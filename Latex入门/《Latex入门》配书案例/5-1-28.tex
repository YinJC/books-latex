% 导言区使用 \usepackage{longtable}
\newcommand\meta[1]{\emph{$\langle$#1$\rangle$}}
\begin{longtable}{|l|l|}
\caption{\texttt{longtable} 环境中的命令汇总} \\
\hline
\endfirsthead
\caption{\texttt{longtable} 环境中的命令汇总(续表)} \\
\hline
\endhead
\hline
\multicolumn{2}{c}{\itshape 接下一页表格……} \\[2ex]
\endfoot
\hline
\endlastfoot
\multicolumn{2}{|c|}{环境的水平对齐可选项} \\ \hline
留空 & 表格居中%
\footnote{实际上,留空的对齐方式是由一组命令控制的,参见宏包文档。} \\
\verb=[c]= & 表格居中 \\
\verb=[l]= & 表格左对齐 \\
\verb=[r]= & 表格右对齐 \\
\hline \multicolumn{2}{|c|}{结束表格一行的命令} \\ \hline
\verb=\\= & 普通的结束一行表格 \\
\verb=\\[=\meta{距离}\verb=]= & 结束一行,并增加额外间距 \\
\verb=\\*= & 结束一行,禁止在此分页 \\
\verb=\kill= & 当前行不输出,只参与宽度计算 \\
\verb=\endhead= & 此命令以上部分是每页的表头 \\
\verb=\endfirsthead= & 此命令以上部分是表格第一页的表头 \\
\verb=\endfoot= & 此命令以上部分是每页的表尾 \\
\verb=\endlastfoot= & 此命令以上部分是表格最后一页的表尾 \\
\hline \multicolumn{2}{|c|}{标题命令} \\ \hline
\verb=\caption{=\meta{标题}\verb=}= & 生成带编号的表格标题 \\
\verb=\caption*{=\meta{标题}\verb=}= & 生成不带编号的表格标题 \\
\hline \multicolumn{2}{|c|}{分页控制} \\ \hline
\verb=\newpage= & 强制分页 \\
\verb=\pagebreak[=\meta{程度}\verb=]= & 允许分页的程度(0--4) \\
\verb=\nopagebreak[=\meta{程度}\verb=]= & 禁止分页的程度(0--4) \\
\hline \multicolumn{2}{|c|}{脚注控制} \\ \hline
\verb=\footnote= & 使用脚注\footnote{普通表格中不能用。},
  注意不能用在表格头尾 \\
\verb=\footnotemark= & 单独产生脚注编号,不能用在表格头尾 \\
\verb=\footnotetext= & 单独产生脚注文字 \\
\hline \multicolumn{2}{|c|}{长度参数} \\ \hline
\verb=\LTleft= & 对齐方式留空时,表格左边的间距,默认为 \verb=\fill= \\
\verb=\LTright= & 对齐方式留空时,表格右边的间距,默认为 \verb=\fill= \\
\verb=\LTpre= & 表格上方间距,默认为 \verb=\bigskipamount= \\
\verb=\LTpost= & 表格下方间距,默认为 \verb=\bigskipamount= \\
\verb=\LTcapwidth= & 表格标题的宽度,默认为 4\,in \\
\end{longtable}
